% hw2a.tex - Comp356 HW 2A Answers

\documentclass{article}
\usepackage{amsmath}
\usepackage{amssymb}

% Define new commands for vectors.
\newcommand{\abs}[1]{\lvert#1\rvert}
\newcommand{\norm}[1]{\lVert#1\rVert}

\title{Comp 356: Computer Graphics\\
Homework 2A: Ray-tracing geometry}
\author{Evan Carmi}
\date{11 October 2010}

\begin{document}
\maketitle
\rule{\textwidth}{.1pt}

\begin{enumerate}
    \item \textbf{Problem:} Assuming the standard camera frame based on the viewpoint, look-at point, and up- direction, compute the camera frame basis expressed as linear combinations of the standard basis.  Your camera frame basis must be orthonormal.

        \textbf{Solution:} First, we must compute the vectors that will compose our basis \(\vec{u}, \vec{v}, \vec{w}\).
        \[\vec{w} \text{ is opposite the view direction so } \vec{w} = \frac{\vec{e} - \vec{P}}{\norm{\vec{e} - \vec{P}}}\]
        \[ \vec{e} - \vec{P} = (4,4,4) - (-1,-1,-1) = (3,3,3) \]
        \[ \vec{w} = \frac{(3,3,3)}{\norm{(3,3,3)}} = \frac{(3,3,3)}{\sqrt{3^2 + 3^2 + 3^2}} = (.577, .577, .577) \]
        \[ \vec{u} = \frac{\vec{up} \times \vec{P}}{\norm{\vec{up} \times \vec{P}}} \text{ and } \vec{up} \times \vec{P} = (0,1,0) \times (.577, .577, .577) \]
        \[= ( (1*.577 - 0*.577), (0*.577 - 0*.577), (0*.577 - 1*.577))  = (.577, 0, -.577)\]
        \[\vec{u} = \frac{(.577, 0, -.577)}{\norm{(.577, 0, -.577)}} = \frac{(.577, 0, -.577)}{\sqrt{.577^2 + 0^2 + -.577^2)}} = (.707, 0, -.707)\]
        \[\vec{u} = \vec{w} \times \vec{u} = (.577, 0, -.577) \times (.707, 0, -.707)




\end{enumerate}
\end{document}
