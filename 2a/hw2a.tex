% hw2a.tex - Comp356 HW 2A Answers

% Declare page size and style.
\documentclass[a4paper, 12pt]{article}
\pagestyle{plain}
\usepackage{fullpage}

% Declare which packages to use..
\usepackage{amsmath}
\usepackage{amssymb}

% Define custom commands for the document.
\newcommand{\abs}[1]{\lvert#1\rvert}
\newcommand{\norm}[1]{\lVert#1\rVert}

% utils.tex - A LaTeX file that contains commonly used commands and other miscellaneous things.
% Usage: to include this file in another .tex file add:
%   % utils.tex - A LaTeX file that contains commonly used commands and other miscellaneous things.
% Usage: to include this file in another .tex file add:
%   % utils.tex - A LaTeX file that contains commonly used commands and other miscellaneous things.
% Usage: to include this file in another .tex file add:
%   \input{./utils.tex}
% to the preamble of that document.

% Define a create title command.
% Based off a header by Matt Adelman.
\newcommand{\HRule}{\noindent\rule{\linewidth}{.5mm}\\}
\newcommand{\hwheader}[1]{\begin{center}
    \textbf{ Evan Carmi\\
    #1\\
    \today
    } \end{center}
    \HRule
}

% to the preamble of that document.

% Define a create title command.
% Based off a header by Matt Adelman.
\newcommand{\HRule}{\noindent\rule{\linewidth}{.5mm}\\}
\newcommand{\hwheader}[1]{\begin{center}
    \textbf{ Evan Carmi\\
    #1\\
    \today
    } \end{center}
    \HRule
}

% to the preamble of that document.

% Define a create title command.
% Based off a header by Matt Adelman.
\newcommand{\HRule}{\noindent\rule{\linewidth}{.5mm}\\}
\newcommand{\hwheader}[1]{\begin{center}
    \textbf{ Evan Carmi\\
    #1\\
    \today
    } \end{center}
    \HRule
}


\begin{document}
\hwheader{Comp 356: Computer Graphics\\Homework 2A: Ray-tracing geometry}

\begin{enumerate}
    \item \textbf{Problem:} Assuming the standard camera frame based on the viewpoint, look-at point, and up- direction, compute the camera frame basis expressed as linear combinations of the standard basis.  Your camera frame basis must be orthonormal.

        \textbf{Solution:} First, we must compute the vectors that will compose our basis \(\vec{u}, \vec{v}, \vec{w}\).
        \[\vec{w} \text{ is opposite the view direction so } \vec{w} = \frac{\vec{e} - \vec{P}}{\norm{\vec{e} - \vec{P}}}\]
        \[ \vec{e} - \vec{P} = (4,4,4) - (-1,-1,-1) = (3,3,3) \]
        \[ \vec{w} = \frac{(3,3,3)}{\norm{(3,3,3)}} = \frac{(3,3,3)}{\sqrt{3^2 + 3^2 + 3^2}} = \fbox{(.577, .577, .577)} \]
        \[ \vec{u} = \frac{\vec{up} \times \vec{P}}{\norm{\vec{up} \times \vec{P}}} \text{ and } \vec{up} \times \vec{P} = (0,1,0) \times (.577, .577, .577) \]
        \[= ( (1*.577 - 0*.577), (0*.577 - 0*.577), (0*.577 - 1*.577))  = (.577, 0, -.577)\]
        \[\vec{u} = \frac{(.577, 0, -.577)}{\norm{(.577, 0, -.577)}} = \frac{(.577, 0, -.577)}{\sqrt{.577^2 + 0^2 + -.577^2)}} = \fbox{(.707, 0, -.707)}\]
        \[\vec{v} = \vec{w} \times \vec{u} = (.577, 0, -.577) \times (.707, 0, -.707)\]
        \[= \left( \left( .577*-.707 \right) - \left( .577*0 \right), \left( .577*.707 \right) - \left( .577*-.707 \right), \left( .577*.707 \right) - \left( .577*.707 \right) \right)\]
        \[ \vec{v} = \fbox{(-.407, .815, -.407)}\]

        These vectors form the orthonormal basis for the camera frame.
        \[\fbox{Basis: $\vec{u}$, $\vec{v}$, $\vec{w}$}\]

    \item \textbf{Problem:} Assuming the standard viewing rectangle that lies in a plane that is orthogonal to \(\vec{w}\) and such that \(\vec{e} - t\vec{w}\) intersects the viewing rectangle in its center, give the coordinates of the lower-left and upper-right corners of the view rectangle in the world frame.

    \textbf{Solution:} 


\end{enumerate}
\end{document}
