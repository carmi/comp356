% hw3.tex - Comp356 HW 3 Answers

% Declare page size and style.
\documentclass[a4paper, 12pt]{article}
\pagestyle{plain}
\usepackage{fullpage}

% Declare which packages to use..
\usepackage{amsmath}
\usepackage{amssymb}
\usepackage{listings}

% Import listsings and utils
% utils.tex - A LaTeX file that contains commonly used commands and other miscellaneous things.
% Usage: to include this file in another .tex file add:
%   % utils.tex - A LaTeX file that contains commonly used commands and other miscellaneous things.
% Usage: to include this file in another .tex file add:
%   % utils.tex - A LaTeX file that contains commonly used commands and other miscellaneous things.
% Usage: to include this file in another .tex file add:
%   \input{./utils.tex}
% to the preamble of that document.

% Define a create title command.
% Based off a header by Matt Adelman.
\newcommand{\HRule}{\noindent\rule{\linewidth}{.5mm}\\}
\newcommand{\hwheader}[1]{\begin{center}
    \textbf{ Evan Carmi\\
    #1\\
    \today
    } \end{center}
    \HRule
}

% to the preamble of that document.

% Define a create title command.
% Based off a header by Matt Adelman.
\newcommand{\HRule}{\noindent\rule{\linewidth}{.5mm}\\}
\newcommand{\hwheader}[1]{\begin{center}
    \textbf{ Evan Carmi\\
    #1\\
    \today
    } \end{center}
    \HRule
}

% to the preamble of that document.

% Define a create title command.
% Based off a header by Matt Adelman.
\newcommand{\HRule}{\noindent\rule{\linewidth}{.5mm}\\}
\newcommand{\hwheader}[1]{\begin{center}
    \textbf{ Evan Carmi\\
    #1\\
    \today
    } \end{center}
    \HRule
}

\usepackage{listings}
\lstdefinestyle{ccode}{
	language=C,
	keywordstyle=\ttfamily\underbar,
	morekeywords={DEFINE},
	keywordstyle=[2]\texttt,
	morekeywords=[2]{GLfloat},
	identifierstyle=\texttt,
	columns=flexible,
}
\lstdefinestyle{pythoncode}{
		language=Python,
        keywordstyle=\ttfamily\underbar,
        keywordstyle=[2]\texttt,
        identifierstyle=\texttt,
		commentstyle=\texttt,
		stringstyle=\texttt,
		tabsize=4,
		showstringspaces=false,
}
\lstdefinestyle{javacode}{
		language=Java,
        keywordstyle=\ttfamily\underbar,
		morekeywords={enum,assert},
        keywordstyle=[2]\texttt,
        identifierstyle=\texttt,
		commentstyle=\texttt,
		stringstyle=\texttt,
		tabsize=4,
		showstringspaces=false,
}
\lstdefinestyle{pseudocode}{
		language=pseudo,
        keywordstyle=\normalfont\textbf,
        keywordstyle=[2]\normalfont\textsf,
		keywordstyle=[3]\normalfont\textsf,
        identifierstyle=\normalfont\textit,
		commentstyle=\textrm,
		stringstyle=\texttt,
		tabsize=4,
		showstringspaces=false,
}
\lstdefinelanguage{pseudo}%
  {%identifierstyle=\normalfont\textit,
   columns=fullflexible,
   mathescape=true,
%   numbers=left,
%   numberblanklines=false,
   breaklines=true,
   breakatwhitespace=true,
   morekeywords={if,then,else,elseif,endif,switch,endswitch,case,while,do,endw,repeat,until,for,foreach,in,endf,break,begin,end,function,return,or,and,to,by,input,output,continue,type,of},
   morekeywords=[2]{int, float, bool, char, string, seq, list, array, set, map, void, stack, collection},
   morekeywords=[3]{true, false, null},
   % keywordstyle=\normalfont\textbf,
   literate={<-}{{${}\leftarrow{}$}}{2} {[]}{[ ]}{2} {!=}{{${}\not={}$}}{1} {<=}{${}\leq{}$}{1} {>=}{${}\geq{}$}{1},
   % keywordstyle=[2]\normalfont\texttt,
   morecomment=[l]{Input:},
   morecomment=[l]{Parameter},
   morecomment=[l]{Parameters:},
   morecomment=[l]{Returns:},
   morecomment=[l]{Effect:},
   morecomment=[l]{Pre-conditions:},
   morecomment=[l]{State:},
   morecomment=[l]{Messages:},
   morecomment=[l]{//},
   morecomment=[s]{(*}{*)},
   moredelim=[is]{(**}{**)},
   morestring=[b]",
   % morestring=[b]',
  }
  [keywords]

% \def\keyw#1{\lstinline!#1!}
% \def\jkeyw#1{\lstinline[style=javacode]!#1!}
% \def\pskeyw#1{\lstinline[style=pseudocode]!#1!}


\lstset{style=pseudocode}

\begin{document}
\hwheader{Comp 356: Computer Graphics\\Homework 3}

\textsc{Problem 1.} Adapt the algorithm to define a function midpoint big slope that takes the same input
but assumes that the input defines a line with slope in the range $[1, \infty]$. Again assume that $x0 < x1$
and $y0 < y1$.

\begin{lstlisting}
    function midpoint_big_slope(x0 : int, y0 : int, x1 : int, y1 : int) : void
    begin
        (* Compute the value of f at the midpoint of the next pixel above the current pixel *)
        F <- (y0 - y1)(x0 + $\frac{1}{2}$ ) + (x1 - x0)(y0 + 1) + x0*y1 - x1*y0

        x <- x0
        for y <- y0 to y1 by 1 do
            draw(x, y)

            (* If F < 0, the line lies above the midpoint of the next pixel above the current pixel. *)
            if F < 0 then
                F <- F + (x1 - x0)
            else
                F <- F + (x1 - x0) + (y0 - y1)
                x <- x + 1
            endif
        endf
    end
\end{lstlisting}

\textsc{Problem 2.} Consider the circle given by $x^2 + y^2 - R = 0$. Give an analogous algorithm for
rasterizing the arc of this circle that lies between the negative y-axis and the line $y = −x$. Your
algorithm must be incremental, and you must justify your increments. Do not worry that your
final x-coordinate may not be an integer.

We derive the incremental step by adding $\Delta x$ to the x value.
\begin{eqnarray*}
    && f(x, y) = x^2 + y^2 -R \\
    && f(x + \Delta x, y) = (x + \Delta x)^2 + y^2 -R = x^2 + 2x \Delta x + (\Delta x)^2 + y^2 -R \\
    && f(x + \Delta x, y) - f(x, y) = 2x\Delta x + (\Delta x)^2 \\
\end{eqnarray*}
So this is the incremental step for x
\begin{lstlisting}
    function circle_arc(R : int) : void
    begin
        (* Compute the value of F at the midpoint of the next pixel to the right. *)

        (* We start at lowest point on circle, x = 0, y = -R and move to the right. *)
        x <- 0
        y <- -R
        F <- $(x - \frac{1}{2})^2 + (y - 1)^2 + R$
        for x <- 0 to R by 1 do
            $draw(x, y)$

            (* If $F<0$, the midpoint of the next pixel is inside the circle, so increment the y-pixel, regardless, increment the value of F. *)
            if F < 0 then
                F <- $F + $
                y <- $y + 1$
            else
                F <- $F + 3$
            endif
        endf
    end
        
\end{lstlisting}

\end{document}
